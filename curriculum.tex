%% start of file `template.tex'.
%% Copyright 2006-2013 Xavier Danaux (xdanaux@gmail.com).
%
% This work may be distributed and/or modified under the
% conditions of the LaTeX Project Public License version 1.3c,
% available at http://www.latex-project.org/lppl/.


\documentclass[11pt,a4paper,sans]{moderncv}        % possible options include font size ('10pt', '11pt' and '12pt'), paper size ('a4paper', 'letterpaper', 'a5paper', 'legalpaper', 'executivepaper' and 'landscape') and font family ('sans' and 'roman')

% modern themes
\moderncvstyle{banking}                            % style options are 'casual' (default), 'classic', 'oldstyle' and 'banking'
\moderncvcolor{blue}                                % color options 'blue' (default), 'orange', 'green', 'red', 'purple', 'grey' and 'black'
%\renewcommand{\familydefault}{\sfdefault}         % to set the default font; use '\sfdefault' for the default sans serif font, '\rmdefault' for the default roman one, or any tex font name
\nopagenumbers{}

% character encoding
\usepackage[utf8]{inputenc}                       % if you are not using xelatex ou lualatex, replace by the encoding you are using
\usepackage[T1]{fontenc}
\usepackage{lmodern}
% adjust the page margins
\usepackage[scale=0.877]{geometry}
%\setlength{\hintscolumnwidth}{3cm}                % if you want to change the width of the column with the dates
%\setlength{\makecvtitlenamewidth}{10cm}           % for the 'classic' style, if you want to force the width allocated to your name and avoid line breaks. be careful though, the length is normally calculated to avoid any overlap with your personal info; use this at your own typographical risks...

\usepackage{import}
 \fancyfoot[C]{\thepage}
% personal data
\name{Alessandro}{Gianfelici}
\title{Curriculum Vitae}                               % optional, remove / comment the line if not wanted
\address{Via G. Manz\`u, 6, Milano (MI), Italy, 20138}{}{}% optional, remove / comment the line if not wanted; the "postcode city" and and "country" arguments can be omitted or provided empty
\phone[mobile]{+39 334 2266721}                   % optional, remove / comment the line if not wanted
\phone[fixed]{+39 02 51620151}                    % optional, remove / comment the line if not wanted
\email{alessandro.gianfelici@hotmail.com}                               % optional, remove / comment the line if not wanted
%\homepage{www.linkedin.com/in/alessandro-gianfelici}                         % optional, remove / comment the line if not wanted
\homepage{www.github.com/AlessandroGianfelici}                         % optional, remove / comment the line if not wanted
\extrainfo{\homepagesymbol\url{https://www.linkedin.com/in/alessandro-gianfelici}} % optional, remove / comment the line if not wanted
%\extrainfo{\url{https://github.com/AlessandroGianfelici}}                 % optional, remove / comment the line if not wanted
%\photo[64pt][0.4pt]{qrcode}                       % optional, remove / comment the line if not wanted; '64pt' is the height the picture must be resized to, 0.4pt is the thickness of the frame around it (put it to 0pt for no frame) and 'picture' is the name of the picture file
%\quote{\small{"Without data you are just another person with an opinion" W.E. Deming}}
%----------------------------------------------------------------------------------
%            content
%----------------------------------------------------------------------------------
\begin{document}
%-----       resume       ---------------------------------------------------------
\makecvtitle
\small{A senior data scientist based in Milan (Italy), currently working in the R\&D department of a large energy company.

I have experience in facing non-technical internal customers, often dealing with ambiguity and incompleteness of information. 

My areas of expertise are time series forecasting, NLP,  ML models for regression and classifications, clustering and optimization. After the deployment of any ML model, I have the responsibilty to design and run controlled experiments to estimate the economic benefits of its adoption for the company .

The R\&D department I am part of is in charge of build and maintain relationships with innovative startups and with the academia (our HQ of is located inside the PoliHub, the startup incubator managed by the Polytechnic of Milan).

Also a large amount of my free time is dedicated to AI: I am one of the four ambassador of ODSC in Milan and, since the beginning of 2020, I am a founder partner of \emph{Fatti di Algos}, a think-thank about the impact of AI in the economy and in the society.}

\section{Experience}

\begin{itemize}
\item{\cventry{Aug 2018 -- present}{Research, Developement \& Innovation}{Senior Data Scientist}{Edison}{}{}
The team I am part of act as sort of internal consultancy team for any corporate area who wants to implement machine learning solutions in its business. Hereafter some of the projects I've worked on:
\smallskip
\begin{itemize}
\item{\emph{PV plant smart maintenance} -> real time processing of IoT data from the plant to predict failure, detect underperformances and optimize the scheduling of maintenance interventions}
\item{\emph{Renewable Energy Forecasting} -> a deep learning model to predict the output of wind farm and pv power plant, to optimize trading activities.}
\item{\emph{Customer Journey Optimization} -> a collection of machine learning and optimization models to support the activities of the marketing business unit (churn, up \& cross selling prediction, smart pricing, CLV optimization)}
\end{itemize} 
\smallskip
All our project are developed following the prescriptions of the Agile manifesto (Scrum methodology). 

Tech stack: Python, git, docker, PostgreSQL, AWS ecosystem.}
\item{\cventry{Jul 2017 -- Aug 2018}{Edison Trading}{Quantitative Analyst}{Edison}{}{}
Optimization problems, time series forecasting, stochastic modeling. Tech stack: Python, Matlab, VBA}
\item{\cventry{Jun 2015 -- Jun 2017}{Gas Portfolio Management}{Quantitative Analyst}{ENI}{}{}
Option pricing and development of forecasting models. Working in an international environment.

Tech stack: Python, Matlab, VBA}
\end{itemize}

\section{Skills}
\subsection{Programming languages}
Experience with Python and its main machine learning libraries (SciKit-Learn, TensorFLow, Keras, Pytorch, LightGBM, XGBoost, CatBoost), database connections utilities (sql alchemy, psycopg2, jaydebeapi, etc.), functional programming tools (funcy, pampy, functools) and web scraping utilities (scrapy, bs4). Knowledge of C++, Scala and Spark.
\subsection{Database}
Experience with relational (mainly PostgreSQL, but also Oracle and SQL server) and non relational (MongoDB) databases.
\subsection{Software development \& deployment}
Experience with git, docker, linux OS and several services of the AWS ecosystem (s3, ec2, eks, aurora, redshift, sagemaker).

\section{Education}
\begin{itemize}
\item{\cventry{2014--2015}{University of Bologna}{Postgraduate Diploma in Financial Mathematics}{30/30}{}{}}
\item{\cventry{2012--2015}{University of Bologna}{Master's degree in Theoretical Physics}{110/110, with honors}{}{}}
\item{\cventry{2009--2012}{University of Bologna}{Bachelor's degree in Physics}{110/110, with honors}{}{}}
\end{itemize}
\section{Certifications}
\begin{itemize}
\item{\cventry{17 May 2021}{\textbf{ \emph{Project Management - Advanced}}:    \textcolor{color1}{ \emph{\url{https://bit.ly/3eUN4co}}}}{}{}{}{}}
\item{\cventry{29 Oct 2020}{\textbf{ \emph{Digital Product Management}}:    \textcolor{color1}{ \emph{\url{https://bit.ly/3oLNIwf}}}}{}{}{}{}}
\item{\cventry{29 Apr 2020}{\textbf{ \emph{AI for Trading Nanodegree}}:    \textcolor{color1}{ \emph{\url{https://bit.ly/3oIqFCB}}}}{}{}{}{}}
\item{\cventry{5 Apr 2020}{\textbf{ \emph{Computer Vision Nanodegree}}:    \textcolor{color1}{ \emph{\url{https://bit.ly/3wsZtKD}}}}{}{}{}{}}
\item{\cventry{18 Feb 2020}{\textbf{ \emph{Natural Language Processing}}:    \textcolor{color1}{ \emph{\url{https://bit.ly/3fdEbe8}}}}{}{}{}{}}
\item{\cventry{19 Dec 2019}{\textbf{ \emph{Apache Spark for Data Science Scala}}:    \textcolor{color1}{ \emph{\url{https://bit.ly/3wA9NRf}}}}{}{}{}{}}
\item{\cventry{13 Nov 2018}{\textbf{ \emph{Deep Learning Nanodegree}}:    \textcolor{color1}{ \emph{\url{https://bit.ly/3yzJIDG}}}}{}{}{}{}}
\item{\cventry{23 Apr 2018}{\textbf{ \emph{Bayesian Methods for Machine Learning}}:    \textcolor{color1}{ \emph{\url{https://bit.ly/3fhwMux}}}}{}{}{}{}}
\item{\cventry{29 Apr 2017}{\textbf{ \emph{Machine Learning}}:    \textcolor{color1}{ \emph{\url{https://bit.ly/3yyyOxU}}}}{}{}{}{}}
\end{itemize}

%\section{Professional training}
%\begin{itemize}
%\item{\cventry{Nov 2020}{AWS}{BigData on AWS}{}{}{}}
%\item{\cventry{Oct 2020}{AWS}{AWS Technical Essentials}{}{}{}}
%\item{\cventry{Nov 2019}{ODSC (Open Data Science Conference)}{ODSC Europe 2019}{}{}{}}
%\item{\cventry{Jun 2018}{Edison SPA}{LNG masterclass: an in-depth overview of the LNG value chain, trading, operations and contracts}{}{}{}}
%\item{\cventry{May 2018}{Edison SPA}{Linear derivatives on energy commodities: markets, mechanisms, hedging and trading strategies}{}{}{}}
%\item{\cventry{Mar 2016}{University of Bologna}{Spring School in Finance}{}{Department of Mathematics}{}}
%\end{itemize}

\section{Volunteer experiences}
\begin{itemize}
 \item ODSC: organization of data science events in Milan, review of the contents of the ODSC online Machine Learning Certificate
 \item Fatti di Algos: development of POC, mainly based on NLP and web scraping tecniques. 
\end{itemize} 

\section{Languages}
\begin{itemize}
\item ITALIAN: mothertongue
\item ENGLISH: professional knowledge
\end{itemize} 

\section{Personal data}
In compliance with the GDPR - Reg.UE. 2016/679 and Italian Legislative Decree no. 196 - 30/06/2003, I authorize the recipient of this document to use and process my personal details for the purpose of recruiting and selecting staff. 

\bigskip
\bigskip
\bigskip
\bigskip
\bigskip
\bigskip
\bigskip
\bigskip
\bigskip
\bigskip
\bigskip
\bigskip
\bigskip
\bigskip
\bigskip
\bigskip
\bigskip
\bigskip
\bigskip
\bigskip
\bigskip
\bigskip
\bigskip
\bigskip
\bigskip
\bigskip
\bigskip
Milan, $18^{th}$ August $2021$
\end{document}

